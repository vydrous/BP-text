% options:
% thesis=B bachelor's thesis
% thesis=M master's thesis
% czech thesis in Czech language
% english thesis in English language
% hidelinks remove colour boxes around hyperlinks

\documentclass[thesis=B,english]{FITthesis}[2012/10/20]

% \usepackage[utf8]{inputenc} % LaTeX source encoded as UTF-8
% \usepackage[latin2]{inputenc} % LaTeX source encoded as ISO-8859-2
% \usepackage[cp1250]{inputenc} % LaTeX source encoded as Windows-1250

\usepackage{graphicx} %graphics files inclusion
% \usepackage{subfig} %subfigures
% \usepackage{amsmath} %advanced maths
% \usepackage{amssymb} %additional math symbols

\usepackage{dirtree} %directory tree visualisation

% % list of acronyms
% \usepackage[acronym,nonumberlist,toc,numberedsection=autolabel]{glossaries}
% \iflanguage{czech}{\renewcommand*{\acronymname}{Seznam pou{\v z}it{\' y}ch zkratek}}{}
% \makeglossaries

% % % % % % % % % % % % % % % % % % % % % % % % % % % % % % 
% EDIT THIS
% % % % % % % % % % % % % % % % % % % % % % % % % % % % % % 

\department{Department of Computer Systems}
\title{Timing Attack on the RSA Cipher}
\authorGN{Martin} %author's given name/names
\authorFN{Andr{\' y}sek}%author's surname
\author{Martin Andr{\' y}sek} %author's name without academic degrees
\authorWithDegrees{Martin Andr{\' y}sek} %author's name with academic degrees
\supervisor{Ing. Ji{\v r}{\' i} Bu{\v c}ek}
\acknowledgements{THANKS (remove entirely in case you do not with to thank anyone)}
\abstractEN{}
\abstractCS{V n{\v e}kolika v{\v e}t{\' a}ch shr{\v n}te obsah a p{\v r}{\' i}nos t{\' e}to pr{\' a}ce v {\v c}esk{\' e}m jazyce.}
\placeForDeclarationOfAuthenticity{Prague}
\keywordsCS{Replace with comma-separated list of keywords in Czech.}
\keywordsEN{RSA, cipher, timming attack}
\declarationOfAuthenticityOption{1} %select as appropriate, according to the desired license (integer 1-6)
% \website{http://site.example/thesis} %optional thesis URL


\begin{document}

% \newacronym{CVUT}{{\v C}VUT}{{\v C}esk{\' e} vysok{\' e} u{\v c}en{\' i} technick{\' e} v Praze}
% \newacronym{FIT}{FIT}{Fakulta informa{\v c}n{\' i}ch technologi{\' i}}

\setsecnumdepth{part}
\chapter{Introduction}



\setsecnumdepth{all}
\chapter{State-of-the-art}

\chapter{RSA}
RSA is public-key cryptosystem which was invented by Ron Rivest, Adi Shamir and Leonard Adleman. The cryptosystem was published in the 1977.

Lorem ipsum dolor sit amet, consectetur adipiscing elit. Fusce eget dignissim odio. Curabitur eu tellus dui. Cras eu malesuada nisl, sit amet sodales est. Pellentesque ullamcorper congue nisl, vel mollis lacus hendrerit eu. Pellentesque vitae mattis orci, quis tempus quam. Donec pulvinar quam congue arcu imperdiet, at vulputate odio interdum. Integer at sodales odio. In gravida diam in imperdiet convallis. Proin et ullamcorper elit. Phasellus eget fringilla augue. Donec id ex mi.

Mauris faucibus facilisis ultricies. Quisque non convallis ex. Nulla facilisi. Fusce eleifend justo eu nibh molestie hendrerit. Vestibulum a mi vel felis scelerisque semper. Sed nec dapibus nunc. Quisque eget leo ut ante pulvinar laoreet. Pellentesque ullamcorper fermentum lobortis.

Integer in malesuada augue, id auctor ante. Aliquam non lectus rutrum, suscipit mauris non, blandit est. Nulla lobortis felis vitae auctor malesuada. Nam tristique libero eros, vitae commodo nisi eleifend vel. In a accumsan lacus. Pellentesque condimentum luctus augue. Nam in ex sed lectus bibendum tempus. Donec ac porttitor purus.

Duis maximus, risus quis ullamcorper pulvinar, enim mi malesuada magna, eu dictum eros velit sed turpis. Integer et elit vestibulum, congue libero vel, vehicula risus. Suspendisse egestas sodales scelerisque. Phasellus eleifend lobortis venenatis. Ut vitae consequat mi, quis condimentum enim. Nullam placerat erat placerat odio porta, a condimentum odio congue. Aliquam erat volutpat. Donec imperdiet metus sodales dolor commodo iaculis.

Integer sed sapien faucibus, placerat lacus maximus, elementum justo. Curabitur tempus velit eget mauris tempor, nec varius diam efficitur. Nullam consequat, erat vitae egestas vehicula, nunc metus tempor velit, et dapibus urna ex consequat mauris. Praesent a sem id enim iaculis lacinia. Nam sagittis neque est, sed elementum lacus aliquet dignissim. Ut vitae odio vel lorem porta malesuada ac eget mi. Duis sit amet tristique justo. Proin aliquam diam a suscipit pulvinar.


\chapter{Analysis}

\chapter{Realisation}
RSA is public-key cryptosystem which was invented by Ron Rivest, Adi Shamir and Leonard Adleman. The cryptosystem was published in the 1977.

Lorem ipsum dolor sit amet, consectetur adipiscing elit. Fusce eget dignissim odio. Curabitur eu tellus dui. Cras eu malesuada nisl, sit amet sodales est. Pellentesque ullamcorper congue nisl, vel mollis lacus hendrerit eu. Pellentesque vitae mattis orci, quis tempus quam. Donec pulvinar quam congue arcu imperdiet, at vulputate odio interdum. Integer at sodales odio. In gravida diam in imperdiet convallis. Proin et ullamcorper elit. Phasellus eget fringilla augue. Donec id ex mi.

Mauris faucibus facilisis ultricies. Quisque non convallis ex. Nulla facilisi. Fusce eleifend justo eu nibh molestie hendrerit. Vestibulum a mi vel felis scelerisque semper. Sed nec dapibus nunc. Quisque eget leo ut ante pulvinar laoreet. Pellentesque ullamcorper fermentum lobortis.

Integer in malesuada augue, id auctor ante. Aliquam non lectus rutrum, suscipit mauris non, blandit est. Nulla lobortis felis vitae auctor malesuada. Nam tristique libero eros, vitae commodo nisi eleifend vel. In a accumsan lacus. Pellentesque condimentum luctus augue. Nam in ex sed lectus bibendum tempus. Donec ac porttitor purus.

Duis maximus, risus quis ullamcorper pulvinar, enim mi malesuada magna, eu dictum eros velit sed turpis. Integer et elit vestibulum, congue libero vel, vehicula risus. Suspendisse egestas sodales scelerisque. Phasellus eleifend lobortis venenatis. Ut vitae consequat mi, quis condimentum enim. Nullam placerat erat placerat odio porta, a condimentum odio congue. Aliquam erat volutpat. Donec imperdiet metus sodales dolor commodo iaculis.

Integer sed sapien faucibus, placerat lacus maximus, elementum justo. Curabitur tempus velit eget mauris tempor, nec varius diam efficitur. Nullam consequat, erat vitae egestas vehicula, nunc metus tempor velit, et dapibus urna ex consequat mauris. Praesent a sem id enim iaculis lacinia. Nam sagittis neque est, sed elementum lacus aliquet dignissim. Ut vitae odio vel lorem porta malesuada ac eget mi. Duis sit amet tristique justo. Proin aliquam diam a suscipit pulvinar.


\setsecnumdepth{part}
\chapter{Conclusion}


\bibliographystyle{iso690}
\bibliography{mybibliographyfile}

\setsecnumdepth{all}
\appendix

\chapter{Acronyms}
% \printglossaries
\begin{description}
	\item[GUI] Graphical user interface
	\item[XML] Extensible markup language
\end{description}


\chapter{Contents of enclosed CD}

%change appropriately

\begin{figure}
	\dirtree{%
		.1 readme.txt\DTcomment{the file with CD contents description}.
		.1 exe\DTcomment{the directory with executables}.
		.1 src\DTcomment{the directory of source codes}.
		.2 wbdcm\DTcomment{implementation sources}.
		.2 thesis\DTcomment{the directory of \LaTeX{} source codes of the thesis}.
		.1 text\DTcomment{the thesis text directory}.
		.2 thesis.pdf\DTcomment{the thesis text in PDF format}.
		.2 thesis.ps\DTcomment{the thesis text in PS format}.
	}
\end{figure}

\end{document}
